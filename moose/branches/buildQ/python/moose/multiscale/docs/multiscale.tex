\section{Introduction}

    We should specify following in XML.

    \begin{enumerate}
    
      \item Morphology of cell. Cell is divided into chemical and electrical
        \textbf{domains}.
      
      \item Domains are more complicated than compartments. They can be a piece
        of volume when cell is considered as it is. These 3-D domains are useful
        in describing local-chemical properties. Domains can also be ``surface''
        enclosing a part of equivalent circuit of the cell. 

      \item A cell is thus divided into domains. When the cell is divided into
        small volumes which are like containers having special localised
        chemistry going on, we call these domains \texttt{chemical domains}.
        When an electrical model of cell is divided into smaller circuits
        enclosed in a surface, we call these sufaces \textbf{electrical domain}.
        A chemical domain need not have one-to-one mapping to electrical domain
        i.e. we need not partition the electrical model of cell according to the volume
        partition of cell. 

      \item A change occurring in chemical domain can influence the electrical
        properties of some electrical domain and vice verse. Therefore, we need
        to specify a mapping between chemical domain and electrical domains.
        This mapping can be bi-directional. Such mapping can be seen as a
        bipartite graph where electrical domains are on the right hand side and
        chemical domains are on the other \footnote{We assume that a change in
          chemical domain does not affect the other chemical domain}. (Note: We
          can store the mapping as a graph in graphml format. Graphml format is
        XML based and networkx library can read it easily.)
    
    \end{enumerate}
   
