% Created 2010-04-28 Wed 14:14
\documentclass[11pt]{article}
\usepackage[utf8]{inputenc}
\usepackage[T1]{fontenc}
\usepackage{graphicx}
\usepackage{longtable}
\usepackage{soul}
\usepackage{hyperref}
\usepackage{chemarr}

\title{KinSynChan}
\author{Subhasis Ray}
\date{28 April 2010}

\begin{document}

\maketitle

\setcounter{tocdepth}{3}
\tableofcontents
\vspace*{1cm}

KinSynChan class is based on the derivations by Destexhe, Mainen and
Sejnowski (1994)\cite{destexhe_efficient_1994}

The model is based on the kinetics of neurotransmitter binding to
receptors at the post synaptic terminal. This can be represented by
the first order kinetic scheme:

\begin{center}
$T + R$
\xrightleftharpoons[{\beta}]{\alpha} 
$TR^{*}$
\end{center}

where the forward rate is $\alpha$ and the backward rate is $\beta$.
This gives the rate of change of the fraction \textit{r} of the
receptors that are open:

$\frac{dr}{dt} = \alpha * [T] * (1 - r ) - \beta * r$

Assuming transmitter concentration in the cleft goes up at time
$t_{0}$ and remains high till time $t1$, the value of $\textit{r}$ at
time t ($t_{0} < t < t_{1}$) is given by:
\begin{equation}
r(t) = r_{\infty} + (r(t_{0}) - r_{\infty}) * e^{-(t-t_{0})/\tau_{r}}
\end{equation}
where 
\begin{center}
  $r_{\infty} = \frac{\alpha * T_{max}}{\alpha * T_{max} + \beta}$
\end{center}

and 
\begin{center}
$\tau_{r} = \frac{1}{\alpha * T_{max} + \beta}$
\end{center}
So the usual time constant for rise-time is $\tau_{r} = \tau_{2}$ in
GENESIS SynChan convention.

When the neurotransmitter-pulse is over, the fraction of open
receptors decays exponentially:
\begin{equation}
  r(t) = r(t1) * e^{-\beta (t - t_{1})}
\end{equation}

where $t_{1}$ is the time when the neurotransmitter pulse gets over.
In MOOSE KinSynChan we call this $pulseWidth$.
Thus $\frac{1}{\beta}$ is the decay time constant $\tau_{1}$ in
GENESIS synchan convention.

From equation 1, 
\begin{center}
  $r'(t) = \frac{r_{\infty}}{\tau_{r}} - \frac{r(t)}{\tau_{r}}$
\end{center}
when compared to the form for exponential Euler method: $y'(t) = A - B
* y(t)$, we see, $A = \frac{r_{\infty}}{tau_{r}}$ , $B =
\frac{1}{tau_{r}}$ and $\frac{A}{B} = r_{\infty}$.

So if we replace these values in the exponential Euler form:
\begin{equation}
  r(t+dt) = r(t) * e^{-\frac{dt}{\tau_{r}}} - r_{\infty} * ( 1 - e^{-\frac{dt}{tau_{r}}})
\end{equation}

This is the equation used in KinSynChan class when there are pending
events in the event queue - which indicates that the current time is
within the pulse duration for those events. Once the transmitter
pulse duration is over, the event is removed from the queue. So, in
that situation, the time evolution of the fraction of open gates
follows equation 2. In this case,
\begin{center}
$r'(t) = - \beta * r(t)$
\end{center}
Taking $tau_{1} = \frac{1}{\beta}$, this becomes
\begin{center}
$r'(t) = - \frac{r(t)}{\tau_{1}}$
\end{center}
So, here $A = 0$ and $B = \frac{1}{\tau_{1}}$ and the difference
equation for $r$ is given by
\begin{equation}
r(t+dt) = r(t) * e^{-\frac{1}{tau_{1}} dt}
\end{equation}

In the MOOSE class, we set $\tau_{r}$ as an alias of the rise time
constant $\tau_{2}$ and the decay time constant $\tau_{1}$ as
described above. Also, the duration of the transmitter pulse, $t_{1}$
is specified by the field $pulseWidth$ in KinSynChan class.

Since $r_{\infty}$, $\tau_{1}$ and $\tau_{2}$ are interdependent, we
do not allow the user to set $\tau_{2}$, but it is calculated from
$\tau_{1}$ and $r_{\infty}$ as $\tau_{2} = ( 1 - r_{\infty}) * \tau_{1}$.

\bibliographystyle{ieeetr}
\begin{thebibliography}{2}
  \bibitem{destexhe_efficient_94}A. Destexhe, Z. F. Mainen and
    T. J. Sejnowski, \emph{An Efficient Method for Computing Synaptic
      Conductances Based on a Kinetic Model of Receptor
      Binding}. Neural Computation, vol. 6, no. 1, pp. 14-18, 1994.
\end{thebibliography}

\end{document}
