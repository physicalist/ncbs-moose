%\section{Introduction}
%
%    \begin{enumerate}
%    
%      \item Morphology of cell. Cell is divided into chemical and electrical
%        \textbf{domains}.
%      
%      \item Domains are more complicated than compartments. They can be a piece
%        of volume when cell is considered as it is. These 3-D domains are useful
%        in describing local-chemical properties. Domains can also be ``surface''
%        enclosing a part of equivalent circuit of the cell. 
%
%      \item A cell is thus divided into domains. When the cell is divided into
%        small volumes which are like containers having special localised
%        chemistry going on, we call these domains \texttt{chemical domains}.
%        When an electrical model of cell is divided into smaller circuits
%        enclosed in a surface, we call these sufaces \textbf{electrical domain}.
%        A chemical domain need not have one-to-one mapping to electrical domain
%        i.e. we need not partition the electrical model of cell according to the volume
%        partition of cell. 
%
%      \item A change occurring in chemical domain can influence the electrical
%        properties of some electrical domain and vice verse. Therefore, we need
%        to specify a mapping between chemical domain and electrical domains.
%        This mapping can be bi-directional. Such mapping can be seen as a
%        bipartite graph where electrical domains are on the right hand side and
%        chemical domains are on the other \footnote{We assume that a change in
%          chemical domain does not affect the other chemical domain}. (Note: We
%          can store the mapping as a graph in graphml format. Graphml format is
%        XML based and networkx library can read it easily.)
%    
%
%    \end{enumerate}
%   
%    \section{Morphology} 



\section{Domains}

A \textbf{domain} is like a compartment of the cell. A cell is made up of
domains. A cell also has a electrical circuit model. A \textbf{chemical domain}
is a compartment of cell with a well-defined chemical model. It may be possible
to classify these domains according to the type of chemical reactions are taking
place in them. We can describe each class in XML and use their instances with
co-ordinates. An \textbf{electrical domain} on the other hand describes the
electrical circuit representation of a compartment of the cell.  A chemical
domain always has a equivalent electrical model or electrical domain. 


\subsection{Representation of domains}

Chemical domains are named \texttt{cd\_<number>} and electrical domain are named
\texttt{ed\_<number>}. Each cell is divided into two or more chemical domains and
equivalent circuit of the cell is divided into two or more electrical domains.

\paragraph{Synaptic activity in chemical domain}

 A chemical domain if it contains synaptic activity, should have an optional
 attribute i.e. \texttt{synaptic\_site = ``yes''}. How would $[ligand]$ change
 with synaptic-activity? A model for such an activity should be embedded in
 xml. 

 
\begin{verbatim}

<cell id=``1`` ... >
    <morphology>
       <domains>
          <domain id=``cd_1'' synaptic_site=``no''>
            <species>list of species </species>
            <reaction id=``reactionA''>
              <reactants> .. </reactants>
              <products> .. </products>
              <rate_coeff> 20 </rate_coeff>
            </reaction>
            <reaction>
            .
            </reaction>
          </domain>
          <domain id=``cd_2'' synaptic_site=``yes''>
            <!-- specify chemical domain here with information about synaptic site-->
          </domain>
          
          <!-- Electrical domains here -->
          <domain id=”ed_1” …>
            <circuit> 
               <!-- spice type netlist or matrix representation or graph -->
               <input> list of points which are input to this domain. </input>
               <output> list of points which are output of this circuit. </output> 
            </circuit>
            <coordinates>
               <!-- coordinates of cell body which this circuit is an equivalent 
                 representation. Does it map over to some chemical domain? -->
            </coordinates>
          </domain>
       </domains>

       <mapping> 
         <!-- see section 1.2 for mapping between the domains -->
       </mapping>
   </morphology>
</cell>
\end{verbatim}

\subsection{Mapping between chemical and electrical domain}

\paragraph{Moving borders of compartment}

    If the borders of the compartment are to move, we need to think of standard
    geometries which can be thought of ``enclosing'' the compartment.

Any change in chemical composition of a chemical domain (Say \texttt{cd\_1}) can
change the electrical properties of an electrical domain (Say \texttt{ed\_3})
(and vice versa?). A user-defined mapping has to be provided in the model.
\footnote{\url{http://graphml.graphdrawing.org/}}.


Such a mapping can be encoded by a graph by the application. XML based graph
representation such as \textbf{graphml} are available and can be used by
application to serialize the graph. Directly writing the mapping in graphml
format will be less user-friendly than having a more readable XML element to
describe such mapping.

How the change in one domain changes properties of other domain are to be
described by \texttt{<mapping>} element. It looks like the following:


\begin{verbatim}
<mapping id=``1'' from=``cd_1'' to=``ed_3''>
    <relation from=``cd_1'' to=``ed_3''>
        <lhs> concCa </lhs>
        <rhs> some_XML_equation_describing_relation_between_[Ca]_and_Ica </rhs>
    </relation>
</mapping>
\end{verbatim}

\paragraph{Format of rhs} Element rhs should be generic enough to describe
mathematical relations. How about using latex math syntax? Some work is needed
to parse and build an equation out of it but it should be straightforward. 


The library python-networkx has good support for graphml format. Each relation
is represented by an edge between two nodes (nodes). We can attach python
objects to edges and nodes. For instance a functor which computes how variation
in $I_{ca}$ changes $[Ca]$ can be attached to the edge. Graphs are great for
non-planner relationships and great many algorithms already exits to analyse
topology.


\section{Adaptors}

Adaptors has to be employed if there is unit-conversion problem. We recommend
that one should stick to the standard units making XML representation unit
agnostic. Writing adaptor needs reference documentation of \textbf{pymoose}.

\section{An example of a neuron}

Let say we have a simple neuron. It has two dendrites connected to soma and one
axon. Each dendrite and axon has 2 chemical domains. 


\input{../models/neuron_with_domain.xml}

